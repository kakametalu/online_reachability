% !TEX root = main_min_disc_dist.tex
\begin{for_journal}
For convenience we define two functions ${U(x):=Z(x)-L}$ and ${h(x):=l(x)-L}$. 
We will show that $U(x)$ is the viscosity solution to a particular time-independent HJ equation
%
\begin{equation}\label{eq:HJ_mdr}
    0 = \min\left\{h(x)-U(x), \max_{u\in\U} \min_{ d\in\D} \!\!\frac{\partial U}{\partial x}(x) f(x,u, d) - \lambda U(x)\right\}.
\end{equation}

We begin by presenting some Lemmas to facilitate the proof.
%
\begin{lemma}
The function $U(x)$ is well defined. 
\end{lemma} 
%
\begin{proof}
Define the sequence $\{U(x,k)\}_k$, where
%
\begin{equation}
U(x,k) = \inf_{\beta[\bu](\cdot) \in \B} \sup_{\bu \in \UU} \inf_{t \in [0,k\Delta t]}(h\big(\bx_{x}^{\bu,\beta[\bu]}(t)\big))\exp(-\lambda  t),
\end{equation}
%
and $\Delta t>0$. The sequence is nonincreasing, since $U(x,k+1) \leq U(x,k)$, and is lower bounded by $-2L$, so it converges. Clearly in the limit this sequence also equals $U(x)$.
\end{proof}



\begin{lemma} \label{dpp}
\emph{Dynamic programming principle.} For $\delta>0$,
%
\begin{equation}
\begin{split} 
U(x)= \inf_{\beta[\bu](\cdot) \in \B} \sup_{\bu \in \UU_\delta} 
\big [
\min\{&\inf_{t \in [0, \delta]} h\big(\bx_{x}^{\bu,\beta[\bu]}(t)\big))\exp(-\lambda  t), 
\\&\exp(-\lambda \delta)U(\bx_{x}^{\bu,\beta[\bu]}(\delta))\}
\big ]
\end{split},
\end{equation}
%
where $\UU_{\delta}$ consists of measurable functions on the interval
$[0,\delta]$.
\end{lemma}

\begin{proof}
Splitting the time interval of the infimum in \eqref{eq:mdr_functional} into $[0,\delta]$ and $t>\delta$, $U(x)$ can be expressed as
%
\begin{equation}
\begin{split}{}{}{}
U(x) =
\inf_{\beta[\bu](\cdot) \in \B} \sup_{\bu \in \UU} 
\big [\min\{&\inf_{t \in [0, \delta]} h\big(\bx_{x}^{\bu,\beta[\bu]}(t)\big))\exp(-\lambda  t),\\ &\inf_{t > \delta} h\big(\bx_{x}^{\bu,\beta[\bu]}(t)\big))\exp(-\lambda  t)\}
\big ]
\end{split}.
\end{equation}
%
Due to time-invariance of the dynamics, if we define $s=t-\delta$, $u_\delta(s)=u(t+\delta)$ and $y=\bx_{x}^{\bu,\beta[\bu]}(\delta)$,
%
\begin{equation}
\begin{split}
U(x) = 
\inf_{\beta[\bu](\cdot) \in \B} \sup_{\bu \in \UU} 
\big [\min\{&\inf_{t \in [0, \delta]} h\big(\bx_{x}^{\bu,\beta[\bu]}(t)\big))\exp(-\lambda  t), \\ &\exp(-\lambda \delta) \inf_{s > 0} h\big(\bx_{y}^{\bu_\delta,\beta[\bu_\delta]}(s)\big))\exp(-\lambda s)\}
\big ]
\end{split}.
\end{equation}
%
The game over the second interval can be optimized independently of the first interval once $y=\bx_{x}^{\bu,\bdelta}(\delta)$ is specified thus it can be replaced with $\exp(-\lambda \delta)U(\bx_{x}^{\bu,\beta[\bu]}(\delta))$. Furthermore $\UU$ is replaced with $\UU_\delta$ since the game is only played explicitly on $[0,\delta]$.
\end{proof}

Before presenting the proof of the viscosity solution, it will be necessary to introduce the concepts of \emph{viscosity subsolution} and \emph{viscosity supersolution}. To simplify notation, we first define the \emph{Hamiltonian} $H:\RR^n \times \RR^n \rightarrow \RR$,
%
\begin{equation}\label{eq:hamiltonian}
H(x,p) = \max_{u\in\U} \min_{ d\in\D} \!\! f(x,u,d)\cdot p.
\end{equation}

\begin{definition} \label{def:sub_sol}
A function $\phi$ (in this case $U$) on $\RR^n$  is a viscosity subsolution of \eqref{eq:HJ_mdr}, if for all $\psi \in C^1(\RR^n)$ and $x_0$ such that $\phi(x_0) = \psi(x_0)$ and $x_0$ attains a local maximum on $\phi- \psi$, then
%
\begin{equation}\label{eq:sub_sol}
    \min\left\{h(x_0)-\psi(x_0), H(x_0,\frac{\partial \psi}{\partial x}) - \lambda \psi(x_0)\right\} \geq 0.
\end{equation}
\end{definition}
%

\begin{definition} \label{def:sup_sol} 
A function $\phi$ (in this case $U$) on $\RR^n$  is a supersolution of \eqref{eq:HJ_mdr}, if for all $\psi \in C^1(\RR^n)$ and $x_0$ such that $\phi(x_0) = \psi(x_0)$ and $x_0$ attains a local minimum on $\phi- \psi$, then 
%
\begin{equation}\label{eq:sup_sol}
    \min\left\{h(x_0)-\psi(x_0), H(x_0,\frac{\partial \psi}{\partial x}) - \lambda \psi(x_0)\right\} \leq 0.
\end{equation}
%
\end{definition}


Now we present the major theoretical result for this section.

\begin{theorem}
The function $U(x)$ is the unique viscosity solution to the time-independent HJ equation given by \eqref{eq:HJ_mdr}.

\end{theorem}

\begin{proof}
The structure of the proof follows the classical approach in \cite{Evans1984}, analogously to \cite{Fisac2015}, and draws from viscosity solution theory. We start by assuming that $U$ is not a viscosity solution and then derive a contradiction to Lemma \ref{dpp}. 

A continuous function is a viscosity solution if it is both a \emph{subsolution} and \emph{supersolution}. Note that $U$ is uniformly continuous due to the continuity assumptions on $f$ and $l$. We first show $U$ is a subsolution of \eqref{eq:HJ_mdr}.

% Subsolution proof
From the local maximum condition in Definition \ref{def:sub_sol} and continuity of system trajectories, there exists a sufficiently small $\delta>0$, such that for $\tau \in [0, \delta]$%
\begin{equation*}
U(\bx_{x_0}^{\bu, \bdelta}(\tau)) \leq \psi(\bx_{x_0}^{\bu, \bdelta}(\tau))
\end{equation*}
%
for all $u(\cdot) \in \UU$ and $d(\cdot) \in \DD$.

For sake of contradiction, assume \eqref{eq:sub_sol} is false, then one of the following must be true
%
\begin{subequations}
\begin{align}
&h(x_0) = \psi(x_0) - \epsilon_1 \label{eq:sub_contra_a}\\
&H(x_0,\frac{\partial \psi}{\partial x}) - \lambda \psi(x_0) = -\epsilon_2 \label{eq:sub_contra_b},
\end{align} 
\end{subequations}
%
for some $\epsilon_1, \epsilon_2 > 0$. If \eqref{eq:sub_contra_a} is true, then 
%
\begin{equation}
h(\bx_{x_0}^{\bu, \bdelta}(\tau))\exp(-\lambda \tau) \leq \psi(x_0) - \frac{\epsilon_1}{2} = U(x_0) - \frac{\epsilon_1}{2}
\end{equation}
%
Incorporating this into the dynamic programming principle (Lemma \ref{dpp}), we have 
%
\begin{equation}
\begin{split}
U(x_0) &\leq \inf_{\beta[\bu](\cdot) \in \B} \sup_{\bu \in \UU_\delta}
\big \{\inf_{t \in [0, \delta]} h\big(\bx_{x}^{\bu,\beta[\bu]}(t)\big))\exp(-\lambda  t) \big \} \\&\leq U(x_0) - \frac{\epsilon_1}{2},
\end{split}
\end{equation}
%
which is a contradiction since $\epsilon_1>0$. Similarly, if \eqref{eq:sub_contra_b}, then for a small enough $\delta>0$ and some nonanticipative strategy $\beta[\cdot]$, 
%
\begin{equation}
H(\bx_{x_0}^{\bu,\beta[\bu]}(\tau), \frac{\partial \psi}{\partial x}) - \lambda \psi(\bx_{x_0}^{\bu,\beta[\bu]}(\tau)) \leq \frac{\epsilon_2}{2}
\end{equation}
%
for all $\tau \in [0,\delta]$ and all inputs $\bu(\cdot) \in \UU$. Due to the $\max{}$ in \eqref{eq:hamiltonian}, for $\tau \in [0,\delta]$
%
\begin{equation}
\begin{split}
&f(\bx_{x_0}^{\bu,\beta[\bu]}(\tau),\bu(\tau), \beta[\bu](\tau)) \cdot \frac{\partial \psi}{\partial x} - \lambda \psi(\bx_{x_0}^{\bu,\beta[\bu]}(\tau)) \leq \\ &H(\bx_{x_0}^{\bu,\beta[\bu]}(\tau), \frac{\partial \psi}{\partial x}) - \lambda \psi(\bx_{x_0}^{\bu,\beta[\bu]}(\tau)).
\end{split}
\end{equation}
%
Combining the two previous inequalities and integrating over the interval $[0,\delta]$ we have
%
\begin{equation}
\exp(-\lambda \delta)\psi(\bx_{x_0}^{\bu,\beta[\bu]}(\delta))-\psi(x_0) \leq \frac{\epsilon_2}{2} \delta
\end{equation}
%
Recalling that $U(x_0)=\psi(x_0)$
%
\begin{equation}
\exp(-\lambda \delta)U(\bx_{x_0}^{\bu,\beta[\bu]}(\delta)) \leq \frac{\epsilon_2}{2} \delta + U(x_0)
\end{equation}
%
Incorporating this into the dynamic programming principle, we have
%
\begin{equation}
U(x_0) \leq \exp(-\lambda \delta)U(\bx_{x_0}^{\bu,\beta[\bu]}(\delta)) \leq \frac{\epsilon_2}{2} \delta + U(x_0),
\end{equation}
%
which is a contradiction, thus we conclude that $U$ is a subsolution.

% Supersolution proof
Next we show that $U$ is a supersolution. From the local minimum condition in Definition \ref{def:sup_sol} and continuity of system trajectories, there exists a sufficiently small $\delta>0$, such that for $\tau \in [0, \delta]$
%
\begin{equation*}
U(\bx_{x_0}^{\bu, \bdelta}(\tau)) \geq \psi(\bx_{x_0}^{\bu, \bdelta}(\tau))
\end{equation*}
%
for all $u(\cdot) \in \UU$ and $d(\cdot) \in \DD$.

If we suppose \eqref{eq:sup_sol} is false, then both of the following must hold 
%
\begin{subequations}
\begin{align}
&h(x_0) = \psi(x_0) + \epsilon_1 \label{eq:sup_contra_a}\\
&H(x_0,\frac{\partial \psi}{\partial x}) - \lambda \psi(x_0) = \epsilon_2 \label{eq:sup_contra_b},
\end{align} 
\end{subequations}
%
for some small $\epsilon_1, \epsilon_2 > 0$. If \eqref{eq:sup_contra_a} is true, then 
%
\begin{equation}
\begin{split}
h(\bx_{x_0}^{\bu, \bdelta}(\tau))\exp(-\lambda \tau) &\geq \psi(x_0) + \frac{\epsilon_1}{2} \\ &= U(x_0) + \frac{\epsilon_1}{2}
\end{split}
\end{equation}
%
Similarly, if \eqref{eq:sub_contra_b}, then for small enough $\delta>0$ and  some input $\bu(\cdot) \in \UU$ 
%
\begin{equation}
H(\bx_{x_0}^{\bu,\beta[\bu]}(\tau), \frac{\partial \psi}{\partial x}) - \lambda \psi(\bx_{x_0}^{\bu,\beta[\bu]}(\tau)) \geq \frac{\epsilon_2}{2}
\end{equation}
%
for all $\tau \in [0,\delta]$ and all nonanticipative strategies $\beta[\cdot]$. 

Due to the $\min{}$ in \eqref{eq:hamiltonian}, for $\tau \in [0,\delta]$
%
\begin{equation}
\begin{split}
&f(\bx_{x_0}^{\bu,\beta[\bu]}(\tau),\bu(\tau), \beta[\bu](\tau)) \cdot \frac{\partial \psi}{\partial x} - \lambda \psi(\bx_{x_0}^{\bu,\beta[\bu]}(\tau)) \geq \\ &H(\bx_{x_0}^{\bu,\beta[\bu]}(\tau), \frac{\partial \psi}{\partial x}) - \lambda \psi(\bx_{x_0}^{\bu,\beta[\bu]}(\tau)).
\end{split}
\end{equation}
%
Combining the two previous inequalities and integrating over the interval $[0,\delta]$ we have
%
\begin{equation}
\exp(-\lambda \delta)\psi(\bx_{x_0}^{\bu,\beta[\bu]}(\delta))-\psi(x_0) \geq \frac{\epsilon_2}{2} \delta
\end{equation}
%
Recalling that $U(x_0)=\psi(x_0)$
%
\begin{equation}
\exp(-\lambda \delta)U(\bx_{x_0}^{\bu,\beta[\bu]}(\delta)) \geq \frac{\epsilon_2}{2} \delta + U(x_0).
\end{equation}
%
Incorporating this into the dynamic programming principle, we have
%

% HELP: How can we split this equation
\begin{equation} 
\begin{split}
U(x) = 
\inf_{\beta[\bu](\cdot) \in \B} \sup_{\bu \in \UU_\delta} 
\big [\min\{ &\inf_{t \in [0, \delta]} h\big(\bx_{x}^{\bu,\beta[\bu]}(t)\big))\exp(-\lambda  t), \\&\exp(-\lambda \delta)U(\bx_{x}^{\bu,\beta[\bu]}(\delta))\}
\big ]\geq
U(x) + \min\{\frac{\epsilon_1}{2}, \frac{\epsilon_2}{2} \delta \},
\end{split}
\end{equation}
%
which is a contradiction, thus $U$ is also a supersolution.

Since we have shown that $U$ is both a viscosity subsolution and viscosity supersolution of the HJ equation, this completes the proof that $U$ is a viscosity solution of \eqref{eq:HJ_mdr}. Uniqueness follows from the classical comparison and uniqueness theorems for viscosity solutions (see Theorem 4.2 in \cite{Barron1989}).
\end{proof}
\end{for_journal}