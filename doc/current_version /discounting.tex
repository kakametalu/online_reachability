% !TEX root = main_min_disc_dist.tex
Inspired by the SDR setting, we now present the MDR optimal control problem. Moving forward any mention of value function $V(x)$ refers to the MR setting, and is defined by \eqref{eq:min_dist_functional}, \eqref{eq:val_function}, and \eqref{eq:HJ_mr}.

\subsection{What to discount?}

A natural proposal for the outcome of the MDR problem would be
 
\begin{equation}
\inf_{t\ge 0}l\big(\bx_{x}^{\bu,\bdelta}(t)\big)\exp(-\lambda  t).
\end{equation}

However, there is an issue with defining this outcome. Recall that discounting makes rewards contribute less to the outcome the further they occur in the future. Since we take an infimum, the discounted reward should become more positive the further it occurs in the future making it less likely to be selected by the infimum. This only works if the reward is nonpositive everywhere, which is not the case for $l$, since it is a clipped signed distance. This is easily fixed with the following outcome for the MDR problem
%
\begin{equation}
\mathcal{Z}\big(x,\bu(\cdot),\bdelta(\cdot)\big) := L + \inf_{t\ge 0}(l\big(\bx_{x}^{\bu,\bdelta}(t)\big)-L)\exp(-\lambda  t).
\end{equation}%
\noindent The quantity in the infimum is always nonpositive since $L$ upperbounds $l(x)$ by construction. Note that if ${\lambda=0}$, i.e. no discounting, then we have the minimum reward outcome \eqref{eq:min_dist_functional}. 

We define the MDR value function as
%
\begin{equation} \label{eq:mdr_val_function}
Z(x):=\inf_{\beta[\bu](\cdot) \in \B} \sup_{\bu \in \UU}\mathcal{Z}\big(x,\bu(\cdot),\beta[\bu](\cdot)\big).
\end{equation}

For convenience we define two functions ${U(x):=Z(x)-L}$ and ${h(x):=l(x)-L}$. 
%
\begin{proposition}\label{prop:visc}
The function $U(x)$ is the unique viscosity solution to the following time-independent HJ equation
\begin{equation} \label{eq:HJ_mdr}
    0 = \min \left\{h(x)-U(x), \max_{u\in\U} \min_{ d\in\D} \!\!\frac{\partial U}{\partial x}(x) f(x,u, d) - \lambda U(x)\right\}.
\end{equation}
\end{proposition}%
\noindent Due to space constraints we omit the proof for Proposition \ref{prop:visc}, which follows the same structure as the viscosity proofs presented in \cite{Evans1984}. 

%Since $L$ is an additive constant, our attention can be focused on the payoff from the second term of equation (\ref{eq:V_lambda}), which yields a related value function
%
%\begin{equation}
%U_{\lambda}(x)=\inf_{\beta[\bu](\cdot) \in \B} \sup_{\bu \in \UU}\mathcal{V}_{\lambda}\big(x,\bu(\cdot),\bdelta(\cdot)\big)-L, 
%\end{equation}
%\noindent thus $V_{\lambda}(x) = U_{\lambda}(x)+L$. It can be shown that the above value function is the viscosity solution to the following variational equality


%% !TEX root = main_min_disc_dist.tex
\begin{for_journal}

For convenience we define two functions ${U(x):=Z(x)-L}$ and ${h(x):=l(x)-L}$. 
We will show that $U(x)$ is the viscosity solution to a particular time-independent HJ equation. We begin by presenting some Lemmas to facilitate the proof.
%
\begin{lemma}
The function $U(x)$ is well defined. 
\end{lemma} 
%
\begin{proof}
Define the sequence $\{U(x,k)\}_k$, where
%
\begin{equation}
U(x,k) = \inf_{\beta[\bu](\cdot) \in \B} \sup_{\bu \in \UU} \inf_{t \in [0,k\Delta t]}(h\big(\bx_{x}^{\bu,\beta[\bu]}(t)\big))\exp(-\lambda  t),
\end{equation}
%
and $\Delta t>0$. The sequence is nonincreasing, since $U(x,k+1) \leq U(x,k)$, and is lower bounded by $-2L$, so it converges. Clearly in the limit this sequence also equals $U(x)$.
\end{proof}



\begin{lemma} \label{dpp}
\emph{Dynamic programming principle.} For $\delta>0$,
%
\begin{equation}
\begin{split} 
U(x)= \inf_{\beta[\bu](\cdot) \in \B} \sup_{\bu \in \UU_\delta} 
\big [
\min\{&\inf_{t \in [0, \delta]} h\big(\bx_{x}^{\bu,\beta[\bu]}(t)\big))\exp(-\lambda  t), 
\\&\exp(-\lambda \delta)U(\bx_{x}^{\bu,\beta[\bu]}(\delta))\}
\big ]
\end{split},
\end{equation}
%
where $\UU_{\delta}$ consists of measurable functions on the interval
$[0,\delta]$.
\end{lemma}

\begin{proof}
Splitting the time interval of the infimum in \eqref{eq:mdr_functional} into $[0,\delta]$ and $t>\delta$, $U(x)$ can be expressed as
%
\begin{equation}
\begin{split}{}{}{}
U(x) =
\inf_{\beta[\bu](\cdot) \in \B} \sup_{\bu \in \UU} 
\big [\min\{&\inf_{t \in [0, \delta]} h\big(\bx_{x}^{\bu,\beta[\bu]}(t)\big))\exp(-\lambda  t),\\ &\inf_{t > \delta} h\big(\bx_{x}^{\bu,\beta[\bu]}(t)\big))\exp(-\lambda  t)\}
\big ]
\end{split}.
\end{equation}
%
Due to time-invariance of the dynamics, if we define $s=t-\delta$, $u_\delta(s)=u(t+\delta)$ and $y=\bx_{x}^{\bu,\beta[\bu]}(\delta)$,
%
\begin{equation}
\begin{split}
U(x) = 
\inf_{\beta[\bu](\cdot) \in \B} \sup_{\bu \in \UU} 
\big [\min\{&\inf_{t \in [0, \delta]} h\big(\bx_{x}^{\bu,\beta[\bu]}(t)\big))\exp(-\lambda  t), \\ &\exp(-\lambda \delta) \inf_{s > 0} h\big(\bx_{y}^{\bu_\delta,\beta[\bu_\delta]}(s)\big))\exp(-\lambda s)\}
\big ]
\end{split}.
\end{equation}
%
The game over the second interval can be optimized independently of the first interval once $y=\bx_{x}^{\bu,\bdelta}(\delta)$ is specified thus it can be replaced with $\exp(-\lambda \delta)U(\bx_{x}^{\bu,\beta[\bu]}(\delta))$. Furthermore $\UU$ is replaced with $\UU_\delta$ since the game is only played explicitly on $[0,\delta]$.
\end{proof}

Now we present the major theoretical result for this section

\begin{theorem}
The function $U(x)$ is the unique viscosity solution to the time-independent HJ equation
%
\begin{equation}\label{eq:HJI_lambda}
    0 = \min\left\{h(x)-U(x), \max_{u\in\U} \min_{ d\in\D} \!\!\frac{\partial U}{\partial x}(x) f(x,u, d) - \lambda U(x)\right\}.
\end{equation}
%
\end{theorem}

\begin{proof}
The structure of the proof follows the classical approach in \cite{Evans1984} and draws from viscosity solution theory. We start by assuming that $U$ is not a viscosity solution and then derive a contradiction to Lemma \ref{dpp}. To simplify notation we introduce the \emph{Hamiltonian}:
\begin{equation}\label{eq:hamiltonian}
H(x,p) = \max_{u\in\U} \min_{ d\in\D} \!\! f(x,u,d)\cdot p
\end{equation}
A continuous function is a viscosity solution if it is both a \emph{subsolution} and \emph{supersolution}. Note that $U$ is uniformly continuous due to the continuity assumptions on $f$ and $l$. We now show $U$ is a subsolution of the VI.

% Subsolution proof
\begin{definition} A function $\phi$ (in this case $U$) on $\RR^n$  is a subsolution, if for all $\psi \in C^1(\RR^n)$ and $x_0$ such that $\phi(x_0) = \psi(x_0)$ and $x_0$ attains a local maximum on $\phi- \psi$, then
%
\begin{equation}\label{eq:sub_sol}
    \min\left\{h(x_0)-\psi(x_0), H(x_0,\frac{\partial \psi}{\partial x}) - \lambda \psi(x_0)\right\} \geq 0
\end{equation}
\end{definition}
%
From the local maximum condition and continuity, we have
%
\begin{equation*}
U(\bx_{x_0}^{\bu, \bdelta}(\delta)) \leq \psi(\bx_{x_0}^{\bu, \bdelta}(\delta))
\end{equation*}
%
for sufficiently small $\delta>0$ and all $u(\cdot) \in \UU$ and $d(\cdot) \in \DD$.

For sake of contradiction, assume \eqref{eq:sub_sol} is false, then one of the following must be true
%
\begin{subequations}
\begin{align}
&h(x_0) = \psi(x_0) - \epsilon_1 \label{eq:sub_contra_a}\\
&H(x_0,\frac{\partial \psi}{\partial x}) - \lambda \psi(x_0) = -\epsilon_2 \label{eq:sub_contra_b},
\end{align} 
\end{subequations}
%
for some $\epsilon_1, \epsilon_2 > 0$. If \eqref{eq:sub_contra_a} is true, then 
%
\begin{equation}
h(\bx_{x_0}^{\bu, \bdelta}(\tau))\exp(-\lambda \tau) \leq \psi(x_0) - \frac{\epsilon_1}{2} = U(x_0) - \frac{\epsilon_1}{2}
\end{equation}
%
Incorporating this into the dynamic programming principle (Lemma \ref{dpp}), we have 
%
\begin{equation}
\begin{split}
U(x_0) &\leq \inf_{\beta[\bu](\cdot) \in \B} \sup_{\bu \in \UU_\delta}
\big \{\inf_{t \in [0, \delta]} h\big(\bx_{x}^{\bu,\beta[\bu]}(t)\big))\exp(-\lambda  t) \big \} \\&\leq U(x_0) - \frac{\epsilon_1}{2},
\end{split}
\end{equation}
%
which is a contradiction since $\epsilon_1>0$. Similarly, if \eqref{eq:sub_contra_b}, then for a small enough $\delta>0$ and some nonanticipative strategy $\beta[\cdot]$, 
%
\begin{equation}
H(\bx_{x_0}^{\bu,\beta[\bu]}(\tau), \frac{\partial \psi}{\partial x}) - \lambda \psi(\bx_{x_0}^{\bu,\beta[\bu]}(\tau)) \leq \frac{\epsilon_2}{2}
\end{equation}
%
for all $\tau \in [0,\delta]$ and all inputs $\bu(\cdot) \in \UU$. Due to the $\max{}$ in \eqref{eq:hamiltonian}, for $\tau \in [0,\delta]$
%
\begin{equation}
\begin{split}
&f(\bx_{x_0}^{\bu,\beta[\bu]}(\tau),\bu(\tau), \beta[\bu](\tau)) \cdot \frac{\partial \psi}{\partial x} - \lambda \psi(\bx_{x_0}^{\bu,\beta[\bu]}(\tau)) \leq \\ &H(\bx_{x_0}^{\bu,\beta[\bu]}(\tau), \frac{\partial \psi}{\partial x}) - \lambda \psi(\bx_{x_0}^{\bu,\beta[\bu]}(\tau)).
\end{split}
\end{equation}
%
Combining the two previous inequalities and integrating over the interval $[0,\delta]$ we have
%
\begin{equation}
\exp(-\lambda \delta)\psi(\bx_{x_0}^{\bu,\beta[\bu]}(\delta))-\psi(x_0) \leq \frac{\epsilon_2}{2} \delta
\end{equation}
%
Recalling that $U(x_0)=\psi(x_0)$
%
\begin{equation}
\exp(-\lambda \delta)U(\bx_{x_0}^{\bu,\beta[\bu]}(\delta)) \leq \frac{\epsilon_2}{2} \delta + U(x_0)
\end{equation}
%
Incorporating this into the dynamic programming principle, we have
%
\begin{equation}
U(x_0) \leq \exp(-\lambda \delta)U(\bx_{x_0}^{\bu,\beta[\bu]}(\delta)) \leq \frac{\epsilon_2}{2} \delta + U(x_0),
\end{equation}
%
which is a contradiction, thus we conclude that $U$ is a subsolution.

% Supersolution proof

Now to show that $U$ is a supersolution. 

\begin{definition} A function $\phi$ (in this case $U$) on $\RR^n$  is a supersolution, if for all $\psi \in C^1(\RR^n)$ and $x_0$ such that $\phi(x_0) = \psi(x_0)$ and $x_0$attains a local minimum on $\phi- \psi$, then by continuity of $l$ and system trajectories, there exists a sufficiently small $\delta>0$, such that for $\tau \in [0, \delta]$
%
\begin{equation}\label{eq:sup_sol}
    \min\left\{h(x_0)-\psi(x_0), H(x_0,\frac{\partial \psi}{\partial x}) - \lambda \psi(x_0)\right\} \leq 0
\end{equation}
%
\end{definition}

From the local minimum condition and continuity, we have
%
\begin{equation*}
U(\bx_{x_0}^{\bu, \bdelta}(\delta)) \geq \psi(\bx_{x_0}^{\bu, \bdelta}(\delta))
\end{equation*}
%
for sufficiently small $\delta>0$ and all $u(\cdot) \in \UU$ and $d(\cdot) \in \DD$.

If we suppose \eqref{eq:sup_sol} is false, then both of the following must hold 
%
\begin{subequations}
\begin{align}
&h(x_0) = \psi(x_0) + \epsilon_1 \label{eq:sup_contra_a}\\
&H(x_0,\frac{\partial \psi}{\partial x}) - \lambda \psi(x_0) = \epsilon_2 \label{eq:sup_contra_b},
\end{align} 
\end{subequations}
%
for some small $\epsilon_1, \epsilon_2 > 0$. If \eqref{eq:sup_contra_a} is true, then 
%
\begin{equation}
\begin{split}
h(\bx_{x_0}^{\bu, \bdelta}(\tau))\exp(-\lambda \tau) &\geq \psi(x_0) + \frac{\epsilon_1}{2} \\ &= U(x_0) + \frac{\epsilon_1}{2}
\end{split}
\end{equation}
%
Similarly, if \eqref{eq:sub_contra_b}, then for small enough $\delta>0$ and  some input $\bu(\cdot) \in \UU$ 
%
\begin{equation}
H(\bx_{x_0}^{\bu,\beta[\bu]}(\tau), \frac{\partial \psi}{\partial x}) - \lambda \psi(\bx_{x_0}^{\bu,\beta[\bu]}(\tau)) \geq \frac{\epsilon_2}{2}
\end{equation}
%
for all $\tau \in [0,\delta]$ and all nonanticipative strategies $\beta[\cdot]$. 

Due to the $\min{}$ in \eqref{eq:hamiltonian}, for $\tau \in [0,\delta]$
%
\begin{equation}
\begin{split}
&f(\bx_{x_0}^{\bu,\beta[\bu]}(\tau),\bu(\tau), \beta[\bu](\tau)) \cdot \frac{\partial \psi}{\partial x} - \lambda \psi(\bx_{x_0}^{\bu,\beta[\bu]}(\tau)) \geq \\ &H(\bx_{x_0}^{\bu,\beta[\bu]}(\tau), \frac{\partial \psi}{\partial x}) - \lambda \psi(\bx_{x_0}^{\bu,\beta[\bu]}(\tau)).
\end{split}
\end{equation}
%
Combining the two previous inequalities and integrating over the interval $[0,\delta]$ we have
%
\begin{equation}
\exp(-\lambda \delta)\psi(\bx_{x_0}^{\bu,\beta[\bu]}(\delta))-\psi(x_0) \geq \frac{\epsilon_2}{2} \delta
\end{equation}
%
Recalling that $U(x_0)=\psi(x_0)$
%
\begin{equation}
\exp(-\lambda \delta)U(\bx_{x_0}^{\bu,\beta[\bu]}(\delta)) \geq \frac{\epsilon_2}{2} \delta + U(x_0).
\end{equation}
%
Incorporating this into the dynamic programming principle, we have
%

% HELP: How can we split this equation
\begin{equation} 
\begin{split}
U(x) = 
\inf_{\beta[\bu](\cdot) \in \B} \sup_{\bu \in \UU_\delta} 
\big [\min\{ &\inf_{t \in [0, \delta]} h\big(\bx_{x}^{\bu,\beta[\bu]}(t)\big))\exp(-\lambda  t), \\&\exp(-\lambda \delta)U(\bx_{x}^{\bu,\beta[\bu]}(\delta))\}
\big ]\geq
U(x) + \min\{\frac{\epsilon_1}{2}, \frac{\epsilon_2}{2} \delta \}
\end{split},
\end{equation}
%

which is a contradiction, thus $U$ is also a supersolution.

Since we have shown that $U$ is both a viscosity subsolution and viscosity supersolution of the variational inequality, this completes the proof that $U$ is a viscosity solution of \eqref{eq:HJI_lambda}. Uniqueness follows from the classical comparison and uniqueness theorems for viscosity solutions (see Theorem 4.2 in \cite{Barron1989}).
\end{proof}
\end{for_journal}





\subsection{Computing the Discounted Value Function}
The discrete approximation of \eqref{eq:HJ_mdr} is given by
%
\begin{equation}\label{eq:U_approx}
    U_{\Delta t} (x) = \min\left\{h(x), \max_{u\in\U} \min_{ d\in\D}  \gamma U_{\Delta t}(x+\Delta tf(x,u,d))\right\},
\end{equation}%
\noindent which can be solved on a grid $G$ via value iteration 
%
\begin{subequations} \label{eq:val_iter_backup}
\begin{align}
&\vec{U}^{0} \in \RR^N_G\\
&\vec{U}^{k+1} = B[U^k]\\
&\vec{U} = \lim_{k\rightarrow \infty} \vec{U}^{k}
\end{align}
\end{subequations}%
\noindent with the backup operator defined as
%
\begin{equation} \label{eq:backup_mdr}
B[\vec{A}] := \min\left\{ \vec{h}, \underset{\pi_u}{\max}\text{ }\underset{ \pi_d}{\min} \gamma P_{\pi_u, \pi_d} \vec{A} \right \}
\end{equation}%
\noindent where $\vec{h}_i = h(x_i)$. The MDR value function $Z(x)$ is then approximated by $I[\vec{U}](x)+L$, where again $I[\vec{U}](\cdot)$ is the interpolation operator. We now prove that \eqref{eq:backup_mdr} is a contraction.

%
\begin{lemma}\label{lem:maxmin} For any two functions $q, g: A \times B \rightarrow \RR$
\begin{equation}
|\max_a \min_b q(a,b) -\max_a \min_b g(a,b)| \leq \max_a \max_b |q(a,b) - g(a,b)|
\end{equation}
\end{lemma}
%
\begin{proof}
Define the minimax optimizers for $q$ as the pair $(a_q,b_q)$, and minimax optimizers of $g$ as the pair $(a_g, b_g)$. Without loss of generality we assume that $q(a_q,b_q) \geq g(a_g,b_g)$.
We then have the following inequalities:
%
\begin{equation*}
\begin{split}
&|\max_a \min_b q(a,b) -\max_a \min_b g(a,b)|
\leq |q(a_q,b_q) - \min_b g(a_q,b)|\\
&\leq |q(a_q,b_{gg}) - g(a_q,b_{gg})| \leq \max_a \max_b |q(a,b) - g(a,b)|
\enspace,
\end{split}
\end{equation*}
with $b_{gg} :=\displaystyle{\arg\min_b g(a_q,b)}$.
%
\end{proof}
%
\begin{proposition} 
The operator given by \eqref{eq:backup_mdr} is a contraction mapping in the infinity norm $|| \cdot ||_{\infty}$ on the space $\RR^{N_G}$.
\end{proposition}
\begin{proof} Defining $B[\cdot]$ as in \eqref{eq:backup_mdr}, take two vectors $A_1, A_2 \in \RR^{N_G}$
\begin{equation*}
\begin{split}
&||B[\vec{A}_1] - B[\vec{A}_2]||_{\infty}=\\
&||\min\left\{ \vec{h}, \underset{\pi_u}{\max}\text{ }\underset{ \pi_d}{\min} \gamma P_{\pi_u, \pi_d} \vec{A}_1 \right \}  - \min\left\{ \vec{h}, \underset{\pi_u}{\max}\text{ }\underset{ \pi_d}{\min} \gamma P_{\pi_u, \pi_d} \vec{A}_2 \right \}||_{\infty}
\end{split}
\end{equation*}%
\noindent Leveraging the identity $\min\{a,b\} = \frac{1}{2}((a+b)- |a-b|)$ and using the shorthand $\Pi[\vec{A}]=\underset{\pi_u}{\max}\text{ }\underset{ \pi_d}{\min} \gamma P_{\pi_u, \pi_d} \vec{A}$ , the above is equal to 
%
\begin{equation*}
\frac{1}{2} ||(\Pi[\vec{A}_1]  - \Pi[\vec{A}_2] ) -  (|\Pi[\vec{A}_1]-\vec{h}|  - |\Pi[\vec{A}_2]-\vec{h}|)||_{\infty},
\end{equation*}%
\noindent which by the triangle inequality, is upper bounded by
%
\begin{equation*}
\frac{1}{2} ||(\Pi[\vec{A}_1]  - \Pi[\vec{A}_2] )||_{\infty} + \frac{1}{2}  ||(|\Pi[\vec{A}_1]-\vec{h}|  - |\Pi[\vec{A}_2]-\vec{h}|)||_{\infty}.
\end{equation*}%
\noindent Given the inequality $|a-b| > |(|a|-|b|)|$, this has upper bound
%
\begin{equation*}
\begin{split}
&||(\Pi[\vec{A}_1]  - \Pi[\vec{A}_2] )||_{\infty}=\\ 
&||\underset{\pi_u}{\max}\text{ }\underset{ \pi_d}{\min} \gamma P_{\pi_u, \pi_d}\vec{A}_1 - \underset{\pi_u}{\max}\text{ }\underset{ \pi_d}{\min} \gamma P_{\pi_u, \pi_d} \vec{A}_2||_{\infty}.
\end{split}
\end{equation*}%
\noindent Finally from Lemma \ref{lem:maxmin}, the last upper bound is 
\begin{equation*}
\underset{\pi_u}{\max}\text{ }\underset{ \pi_d}{\max} ||\gamma P_{\pi_u, \pi_d} (\vec{A}_1 - \vec{A}_2)||_{\infty} \leq \gamma||\vec{A}_1 - \vec{A}_2||_{\infty},
\end{equation*}%
\noindent where the last inequality comes from the fact that $P_{\pi_u, \pi_d}$ is a stochastic matrix for all policies, thus $||P_{\pi_u, \pi_d}||_{\infty} = 1$.
\end{proof}



\subsection{Under- and Over-Approximating the Reachable Set}
With MDR formulation there is no particular level curve of the value function that characterizes the reachable set. However, it is possible to find level curves that correspond to over and under approximations of the reachable set. 

We have the inequality $Z(x) \geq V(x)$ because the terms being discounted in the outcome are nonpositive. It immediately follows that
%
\begin{equation} \label{eq:reach_set}
\{x \mid V_{\lambda}(x) \le 0\} \subseteq \R(\T),
\end{equation}  

For an over-approximation we first need to characterize the error between $Z(x)$ and $V(x)$. The difference between the two functions can be bounded. Define $\tau(x)$ as the time when the minimum distance to the target is achieved for a trajectory starting at state $x$ under the optimal control and disturbance signals. Then we have the following bound
%
\begin{equation}
Z(x) - V(x)  \leq (L - l(\bx_{x}^{\bu,\bdelta}(\tau(x))))( 1 -  \exp(-\lambda \tau(x))) 
\end{equation}%
\noindent Noting that $V(x)=l(\bx_{x}^{\bu,\bdelta}(\tau(x)))$, we get the resulting inequality
%
\begin{equation} \label{eq:val_error}
Z(x) -  V(x) \exp(-\lambda \tau(x)) \leq L( 1 -  \exp(-\lambda \tau(x))), 
\end{equation}%
\noindent Furthermore, outside the reachable set $V(x)>0$ leading to
%
\begin{equation}
Z(x) -  V(x)  \leq L( 1 -  \exp(-\lambda \tau(x))) \quad \forall x \not\in \R(\T).
\end{equation}

Assuming an upper bound  ${\bar{\tau} \geq \tau(x)}$, we have the following over-approximation for the reachable set
%
\begin{equation} \label{eq:reach_set}
\R(\T) \subseteq  \{x \mid Z(x) \le L( 1 -  \exp(-\lambda \bar{\tau})) \}.
\end{equation} 

It is clear from \eqref{eq:val_error} that the tightness of the approximations can be tuned via the discount rate $\lambda$.
